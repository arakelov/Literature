

\maketitle
\tableofcontents
\break
\newpage

\newpage

\chapter{Группы}
\section{Алгебра, полугруппа, моноид}
{

\begin{Def}

Бинарной алгеброй называется непустое множество $S$ произвольной природы, с заданной на нем бинарной операцией $\beta : S^2 \rightarrow S $.
\end{Def}
Если операция $\beta$ ассоциативна то бинарная алгебра называется \textbf{полугруппой}. Для удобства будем вместо $\beta(a,b)$ писать просто $ab$. Левым единичным элементом \textbf{$e$}, или просто  леовой единицей, называется элемент удовлетворяющий следующему свойству: 
 \begin{center}
  \fbox {$\forall(a\in S)$  $ea=a $}
  \end{center} 
Аналогично вводится понятие правой единицы - элемента удовлетворяющего свойству: 
    \begin{center}
  \fbox {$\forall(a\in S)$  $ae=a $}
  \end{center}
Заметим, что если в полугруппе имеется и левая $e_l$ и правая единица $e_r$, то они совпадают.
$$ e_r=e_l e_r=e_l  $$

\begin{Def}
\textbf{Моноидом} называется полугруппа с левой и правой единицей.\end{Def}
Более точно, моноид это пара, состоящая из некоторого непустого множество произвольной природы $S$ и  заданной на нем бинарной операцией $\beta, \beta(x,y)=xy$, для которой справедливы следующие свойства:
\begin{center}
M1: $x(yz)=(xy)z;$ \textbf{ассоциативность};\\
M2: $(\exists e \in S):(\forall x \in S)  ex=xe=x;$  \textbf{наличие единицы.}
\end{center}
}
\section{Понятие группы}
{
\begin{Def}
Непустое множество произвольной природы $ \mathfrak {B}$ (например чисел, отображений, матриц) называется \textbf{группой}, если выполняются следующие условия:

\par \textbf{1. Задан закон композиции,} который каждой паре элементов $(a,b)$ из $\mathfrak {B}$ ставит в соответствие третий элемент $c$ из того же множества, как правило называемый произведением элементов $a$ и $b$, и обозначаемый как $ab$ или $a \cdot b$.

\par \textbf{2. Закон ассоцитивности.} Для любых элементов $a,b,c $ из $\mathfrak {B}$ 
\begin{center}
\fbox{$a(bc)=(ab)c.$}
\end{center}

\par \textbf{3.} В $\mathfrak {B}$, отностительно заданного закона композиции, существует (левая)единица, т.е. элемент удовлетворяющий свойству 
\begin{center}

\fbox{$ea=a,$ для всех $a$ из G.}
\end{center}

\par \textbf{4.} Для каждого элемента $a$ из $\mathfrak {B}$, отностительно заданного закона композиции, существует хотя бы один(левый) $a^{-1}$ обратный элемент, определяемый свойством: 
\begin{center}

\fbox{ $ a^{-1} \cdot a=e $}
\end{center}
\end{Def}
Стоит заметить, что произведение элементов в группе может зависеть от порядка следования сомножителей, и не всегда $a\cdot b= b\cdot a.$
Группа называется конечной, если ее множество  содержит конечное число элементов, иначе будем говорить, что имеем дело с бесконечной группой.
\\ \\
\textbf{Примеры}

\textit{1. Возьмем в качестве множества элементов целые числа $Z$. В качестве закона композиции будем рассматривать простое умножение. Проверим будет ли являться группой пара $[Z,\cdot^2]$}\\
Условия 1 и 2 выполняются, так как умножение целых чисел является ассоциативной операцией. Условие 3 и 4 в данном случае выполняться не будут, так как не существует единичного элемента относительно умножения в множестве $Z$. Таким образом $[Z, \cdot^2]$- группой не является.\\
Если в качестве множества мы рассмотрим целые числа без нуля $Z/0$, то тогда в качестве единичного элемента можно взять $1$ и условие 3 будет выполняться. Однако даже при таком подходе мы не получим группу, так как отноститльно умножения для всех элементов кроме $1$ не будет существовать обратного.

\textit{2. Возьмем в качестве множества опять целые числа $Z$, а качестве операции - сложение чисел.}\\
В этом случае условия 1 и 2 опять же выполненны. В качестве единичного элемента возьмем $0$, т.е. условие 3 тоже выполненно. В качестве обратного элемента для любого элемента $a$ досточно взять элемент $-a$, т.к. $a+(-a)=0$. Мы получили выполнение всех 4-х условий, поэтому пара $[Z, \cdot^2]$ является группой.

\textit{3. Если в качестве множества взять множество состоящее только из единицы, а в качестве закона композиции рассматривать обычное умножение, то мы опять получим группу $[\{1\},\cdot^2]$.}

В первых двух примерах мы имеем дело с бесконечными группами. В примере 3, построенная группа является конечной.
\begin{Def}
Группа называется \textbf{абелевой}, если в ней выполняется закон коммутативности: $ab=ba$ для всех $a$ и $b$ из заданного множества.
\end{Def}
Докажем несколько простых лемм, которые понадобяться нам в дальнейшем.
\newtheorem{Lem}{Лемма}
\begin{Thr}
В каждой группе, для любого элемента \textbf{$a$}, его правый обратный и левый обратный совпадают.
\end{Thr} 
\begin{proof}
$$a^{-1}aa^{-1}=ea^{-1}=a^{-1}$$
Домножим левую и правую части на элемент обратный к $a^{-1}$.Получим:
$$aa^{-1}=e$$
Из последнего следует доказательство леммы.
\end{proof}
\begin{Lem}
Для элемента $a^{-1}$ обратным элементом является $a$.
\end{Lem}
\begin{proof}
Пусть $x$ обратный элемент к $a^{-1}$. Тогда имеем:
$$a^{-1}x=e$$
Домножим левую и правую части уравнения на $a$ и получим:
$$ex=ae$$
$$x=a$$
\end{proof}
\begin{Lem}
Каждая левая и правая единицы совпадают.
\end{Lem}
\begin{proof}
$$ae=aa^{-1}a=ea=a$$
\end{proof}
Заметим, что уравнения $ax=b$ и $ya=b$ разрешимы. А именно для первого случая $x=a^{-1}b$, а для второго $y=ba^{-1}.$ Так как:
$$a(a^{-1}b)=(aa^{-1})b=b$$
$$(ba^{-1})a=b(aa^{-1})=b$$

\begin{Lem}
Обратным элементом к произведению $(ab)$, является $b^{-1}a^{-1}$, т.е. $(ab)^{-1}=b^{-1}a^{-1}$
\end{Lem}
\begin{proof}
Пусть $x$ обратный элемент к $(ab)$. Докажем что $x=b^{-1}a^{-1}$. \\По условию:
$$ (ab)x=e$$
Умножим левую и правую части на $b^{-1}a^{-1}$, получим:
$$b^{-1}a^{-1}abx=b^{-1}a^{-1}e$$
$$b^{-1}bx=b^{-1}a^{-1}$$
$$x=b^{-1}a^{-1}$$
\end{proof}
\subsection{Аддитивные группы}
В определение группы, использование в качестве операции-умножения, не является обязательным. Вместо умножения $\cdot^2$, в качестве операции, может использоваться простое сложение $+^2$. В этом случае  , обычно, единичный элемент обозначается как $0$, а сама группа называется аддитивной или модулем.

В аддитивных группах, обычно пологают что сложение коммутативная операция, т.е.
$$a+b=b+a.$$
Обратный к $a$ элемент в аддитивных группах обозначается как $-a$, и вместо $a+(-b)$ обычно пишут $a-b$.\\ \\
\textbf{Примеры}

\textit{1. Примером аддитивной группы служит множество целых чисел со сложением и нулем в качестве единичного элемента.}

\textit{2. Аддитивной группой также является множество $n$-мерных векторов, с введенным на нем операцией покоординатного сложения.}
\subsection{Подстановки}
\subsection{Подгруппы}
Понятие подгруппы широко используется в теории групп. Многие прикладные задачи, построенны на свойстваъ подгрупп. В этом разделе мы рассмотрим понятие подгруппы и основные свойства, которыми обладают подгруппы группы.

Формально, пусть у нас имеется некторая группа $\mathfrak {B}$. Подгруппой в данной группе будет группа, множество элементов которого является подмножеством группы $\mathfrak {B}$.
Определим понятие подгруппы более строго.
\begin{Def}
Подмножество $\mathfrak {b} $группы $\mathfrak {B}$ называется подгруппой если выполняются следующие условия:

1. $\forall x \in \mathfrak {b}, \forall y \in \mathfrak {b}, xy \in \mathfrak {b} $

2. $ \forall x \in \mathfrak {B}, x^{-1} \in \mathfrak {B}.$ 
\end{Def}

Первое свойство требует, чтобы вместе с любыми двумя элементами $x,y$ в подгруппе содержалось и их произведение.
Второе свойство требует, чтобы для каждого жлемента подгруппа содержала и обратный к нему элемент.

При выполнении данных двух свойств, подгруппа $\mathfrak {b}$ будет сново являться группой. Это очевидно, так как если аксиомы группы 1-2 выполняются в группе $\mathfrak {B}$, то они выполняются и в подгруппе $\mathfrak {b}$. Выполнение акиомы группы 4 следует из свойства 2, т.к. $aa^{-1}=e \in \mathfrak {b}.$ 

Пусть $a,b,c,... $  элементы некторой группы $\mathfrak {B}$ $\mathfrak {B}$ могут существать подгруппы, содержащие данные элементы. В этом случае пересечение подгрупп снова явялется подгруппой в данной группе. Сформулируем более сильную теорему.

\begin{Lem}
Пересечение, любого количества подгупп, группы $\mathfrak {B}$ снова является подгруппой в $\mathfrak {B}$.
\end{Lem}
\begin{proof}
Пусть у нас имееется любое количество подгрупп $\mathfrak {a}, \mathfrak {b}, \mathfrak {c}...$ группы $\mathfrak {B}$. И пусть их пересечение $\mathfrak {d}$ содержит элементы $a,b,c,...$.
Докажем что $\mathfrak {b}$ снова является подгруппой.

Рассмотрим произведение любых двух элементов $ab$. т.к. $a \in \mathfrak {a}$ отсюда следует, что $ab \in \mathfrak {a}$. Но $a,b$ также входят и $\mathfrak {b}$ и в остальные подгруппы. Отсюда следует, что и $ab \mathfrak {b}, ab \in \mathfrak {c}...$. Это означает что, $ab \in {d}$. То есть, этим доказывается выполнение условия 1 подгруппы. Аналогично доказывается, тот факт, что вместе с каждым элементов, множество $\mathfrak {d}$ содержит и обратный к нему. Это означает, что в множестве $\mathfrak {d}$ выполняются условия подгруппы, а это означает, что $\mathfrak {d}$- подгруппа.    
\end{proof}
}
\section{Кольца, тела, поля}
{
\subsection{Кольца}
Алгебра и арифметика оперируют элементами различной природы. Это могут быть числа, матрицы, перестановки, отображения и т.д.
В этой главе мы рассмотрим еще одну абстрактную структуру. 

Под системой с двойной композицией, подразумевается произвольное множество элементов $a,b,c,d,...$, для которых однозначно определенны две операции, обычно называемые сложением $+ $ и умножением $*$.
}
\chapter{Элементы теории чисел}
\section{Делимость и делители}

Одним из ключевых понятий в тоерии чисел является понятие деления одного числа на другое. Пока, мы будем считать что находимся в поле целых чисел, и если понадобится, то будем расширять данное поле. Основные теоремы и свойства, которые мы покажем для целых чисел, легко обобщаются на многие расширения поля целых чисел. Мы будем говорить что $a$ делит $b$, если для некторого $k$ выполняется соотношение $ka=b$. 
Факт деления $b$ на $a$ будем обозначать следующим образом: $a|b$.
Заметим, что для любого числа $a$ существуют, так называемые, тривиальрные делители: \textbf{$a, 1$}.
\section{Простые и составные числа}
Целые числа, среди делителей которых только тривиальные делители, называются простыми. Простые числа обладают многими замечательными свойствами и играют важнейшую роль в прикладной алгебре и в теории чисел. Многие криптографичесие системы основанны именно на свойствах таких чисел.
Приведем пример простых чисел:
$$\ldots -5, -3, -1, 1, 2, 3, 5, 7, 11, 13, 17, 23, 29 \ldots$$
Ученные-математики издавно заметили красивые особенности таких чисел. Большинство существующих шифровальных алгоритмов основанно на том, что не существует быстрого способа, который по некторому числу, смог определить является оно простым или нет с \textbf{абсолютной точностью}. Здесь подчеркивается, абсолютная точность, потому, что существуют различные вероятностные алгоритмы проверки на простоту. Это такие алгоритмы, которые получая на вход некторое число $n$ могут с некторой вероятностью $P$ утверждать, что $n -$ простое. почти всегда, вероятность напрямую зависит от времени работы и от количества проделанных итераций алгоритма. Обычно чем больше итераций мы проведем, тем с большей вероятностью можем утверждать что данное число простое. Мы рассмотрим такие алгоритмы в следующих главах.

\begin{Lem}
Простых чисел бесконечно много.
\end{Lem}
\begin{proof}
Докажем от противного.

Допустим, что множество простых чисел-конечно. Тогда существует наибольшее из них. Обозначим его $n$.

Рассмотрим число следующего вида:

$$p=1\cdot 2 \cdot 3 \cdot 4 \cdot \ldots \cdot n = n!$$
Число $p$ - это произведение всех чисел от $1$ до $n$. Докажем что $p+1$-простое.
Так как, $2$ | $p$ $\rightarrow 2 \nmid (p+1)$. Аналогично можно доказать, что $p+1$ не делится ни на какое другое число от $2$ до $p$, из чего мы можем сделать вывод о том, что $p+1$- простое число. Простота  $p+1$ противоречит нашему предположению, поэтому простых чисел бесконечно много.  
\end{proof}

\section{Деление и остатки}
Относительно заданного числа $n$ все целые числа можно разбить на две группы- те которые кратны $n$, т.е. делятся на $n$, и те которые не делятся на $n$ без остатка. Большая часть теории чисел основанная на разделении последней группы на классы эквивалентности, в зависимости от того, что получается в остатке при делении на $n$. 

Данное разбиение основанно на следующей теореме.
\begin{Lem}[О делении]
Для любого целого числа $a$ и любого положительного целого $n$, существует единственная пара целых чисел $q$ и $r$, таких, что $0\leq r < n$ и $a=qn+r$. 
\end{Lem}
Доказательство данной теоремы довольно тривиально, поэтому здесь она приведенная без него.
Величина $q=\lfloor a/n \rfloor$ называется \textbf{частным} деления. Величина $r=a\ mod\ n$ называется \textbf{остатком от деления}. Таким образом, $n | a$, тогда и только тогда когда $a\ mod\ n=0$.

В зависимости от того, чтому равны остатки чисел от деления на $n$ (\textit{модули по $n$}), их можно разбить на $n$ классов эквивалентности.
Класс эквивалентности по модулю $\textbf{n}$, в котором содержится целое число $a$ имеет следующий вид:
$$ [a]_n=\{a+kn :k \in Z\}$$

Запись $a \in [b]_n$ означает, что  $a\ \equiv \ b \ ( mod\ n)$
Множество всех таких классво эквивалентности имеет вид:
$$ \textbf{Z}_n={[a]_n: 0\leq a\leq n-1}$$

\section{Общие делители и наибольшие общие делители}
{
Число $c$ называется общим делителем чисел $a,b$, если $c$ делит одновременно и $a$, и $b$. Для любых двух чисел 1 является их общим делителем. 
Например для чисел $15$ и $6$ общими делителями служат $1,3$.
Важное свойство общих делителй заключается в том, что для всех целых чисел:
\begin{equation}
\label{a|b&a|c=>a|b+c}
\textit{из}\ d\ |\ a\ \textit{и}\ d\ |\ b\ \textit{следует, что }\ d\ |\ (ax+by)
\end{equation}
Максимальное из общих делителей двух чисел называется - их наибольим общим делителем. Мы будем обозначать его $gcd(a,b)$.
В приведенном выше примере $gcd(15,6)=3$. Понятие наибольшего общего делителя является одним из основных в теории чисел и многие вещи основанны на свойствах наибольшего общего делителя.
Приведем некоторые свйоства наибольшего общего делителя двух чисел.


\begin{equation}
\label{gcd(a,b)=gcd(b,a)}
gcd(a,b)=gcd(b,a)
\end{equation}
\begin{equation}
\label{gcd(a,b)=gcd(-a,b)}
gcd(a,b)=gcd(-a,b)
\end{equation}
\begin{equation}
\label{gcd(a,b)=gcd(|a|,|b|)}
gcd(a,b)=gcd(|a|,|b|)
\end{equation}
\begin{equation}
\label{gcd(a,0)=a}
gcd(a,b)=|a|
\end{equation}
\begin{equation}
\label{gcd(a,ka)=a}
gcd(a,ka)=a \forall k \in \textbf{Z}
\end{equation}
}

Сформулированная ниже теорема является довольно полезной и мы будем часто на нее ссылаться.

\begin{Thr}
Если $a$ и $b$ произвольные целые числа, отличные от нуля, то величина $gcd(a,b)$ равна наименьшему положительному элементу множества $\{ax+by:x,y \in \textbf{Z}\}$
\end{Thr}
\begin{proof}
Обозначим через $s$ наименьшую положительную линейную комбинацию чисел $a$ и $b$, т.е. $s=ax+by$ для некоторых $x,y \in \textbf{Z}$.
Пусть $q=\lfloor a /s \rfloor$. Тогда имеем:
$$
a \bmod s = a - qs = a - q(ax + by) = a(1 - qx) + b(-qy),
$$
поэтому величина  $a \bmod s$ также является линейной комбинацией чисел $a,b$.  Имеет место соотношение $$ 0 \leq a \bmod s \leq s.$$
Но поскольку, $s$-наименьшая из таких комбинаций, отсюда следует что $a \bmod s = 0$. Это означает, что $s | a$. Аналогично можно доказать и для $b$. Тем самым, мы показали что $s$ является общим делителем $a$ и $b$, т.е. $s | a$ и $s | b$. Справедливо равенство $$ gcd(a,b) \geq s.$$ 
Из \eqref{a|b&a|c=>a|b+c} следует что $$gcd(a,b) | s,$$
так как, $s$ линейная комбинация $a,b$. Из последнего следует, что $$gcd (a,b) \leq s$$
Объединяя два соотнощения 
$$ 
gcd (a,b) \leq s \textit{и}  gcd(a,b) \geq s
$$
делаем вывод, что $$s= gcd(a,b).$$
\end{proof}
\begin{Cons}
Для любых целых чисел $a$ и $b$  и произвольного неотрицательного числа $n$ справедливо соотношение:
$$ gcd(an,bn)=n gcd(a,b)$$
$$ d | gcd(a,b)$$ 
\end{Cons}

\begin{Cons}
Для всех положительных чисел $a,b,n$, из условия что $n|ab$ $gcd(a,n)=1$ следует соотношение $n|b.$ 
\end{Cons}
\begin{Cons}
Для любых целых чисел $a$ и $b$ из соотношение $d|a$ и $d|b$ следует, что 
$$ d | gcd(a,b)$$ 
\end{Cons}

Докажем еще одну важнейшую теорему, сформулированную в XVII веке Пьером Ферма и играющую одну из ключевых роле в теории чисел.
\begin{Thr}[Малая теорема \textbf{П.Ферма}]
Пусть $a,p$ - произвольные взаимно простые числа. Тогда, если $p$ - простое, то справедливо сравнение:
\begin{equation} \label{mPherma}
a^{p-1}\equiv 1 \pmod p
\end{equation}

\end{Thr}
\begin{proof}
Существуют разные подходы к доказательству данной теоремы. Мы рассмотри наиболее изящное и простое доказательство, основанное на теории групп и на теореме \textbf{Лагранжа}.

Пусть $\mathfrak{G}$ - конечная группа порядка $n$. Тогда По теореме \textbf{Лагранжа}, из того что порядок элемента $g \in \mathfrak{G}$ делит порядок группы, следует что $g^n=e$.
Рассмотрим группу вычетов по модулю $ p- Z_p $. Порядок данной группы - $p$. Ненулевые элементы $Z_p$ образуют группу по умножению-${Z_p}^*.$
Порядок ${Z_p}^*$ очевидно, равен $p-1$. В данной групее порядок любого элемента, является делителем порядка группы, т.е. $p-1$. В итоге получаем что для всех элементов $k \in {Z_p}^*$,
$k^{p-1}=e$. Из последнего вытекает доказательство теоремы.

\end{proof}

\chapter{Теория кодирования}
\section*{Введение}
В данной главе мы рассмотрим некоторые проблемы из теории передачи информации, а именно двоичное кодирование и декодирование сигналов, передаваемых по некоторому каналу с шумом.
Типичная ситуация следующая: у нас есть последовательность символов, конечной длины, из некторого алфавита. Мы хотим передать данную последовательность по некоторому каналу с шумом и с ненулевой вероятностью $q$, каждый передаваемый символ будет принят ошибочно. Допустим, что мы передаем последовательность длины $ 10000 $знаков и $q=0.01\% $. Даже при, такой, относительно небольшой вероятности ошибки, вероятность $ P_0 $ того, что наша последовательность, при прямой передачи символа за символом, будет передана абсолютно правильно будет следующей: 
$$P_0=(1-0.01)^{10000}\simeq 10^{-4.4}<0.004\% $$   
Данный результат вытекает из классической формулы Бернули, которую можно найти в любом учебнике по теории вероятностей.
В дальнейшем, в целях удобства, мы будем пердполагать, что наш алфавит двоичный и состоит из двух символов $$\Sigma=\{0,1\}$$.
Все изложенное далее можно обобщить и на любой другой алфавит, содержащий произвольной количество элементов.
\section{Кодирование}
Во многих системах передачи информации, за ошибку даже в одном бите приходится дорого платить. Поэтому одной из главных задач, теории передачи информации является уменьшение вероятности искажения передаваемых данных.
В этой главе мы рассмотрим эффективные методы увеличения надежности передачи информации, с помощью систематических кодов разного типа. Большая их часть принадлежит к классу \textit{групповых} кодов, и  основывается на теореме \textit{Лагранжа}. 

Идея, положенная в основу всех систематических кодов следующая:
последовательности, подлежащие передачи, кодируются последовательностями большей длины. Приемник, на основе дополнительной информации, способен распознавать или исправлять ошибки, вызванные шумом.
Принятая последовательность декодируется по определенной схеме в изначальную последовательность символов до кодирования.
\begin{Def}
Двоичным $(m,n)$-кодом, называется пара, состоящая из схемы кодирования: $$E:2^m\rightarrow 2^n,$$
и схемы декодирования:
$$D:2^n\rightarrow 2^m,$$
где $2^n$-это множество всех двоичных последовательностей длины $n$. 
\end{Def}
Функции $E и D$ выбираются так, чтобы функция $H=E\diamond H\diamond D$, где $H$- функция ошибок, с вероятностью близкой к единице была тождественной.

Все коды можно разделить на два класса:\\ \textbf{Коды с обнаружением ошибок} и \textbf{Коды с исправлением ошибок.}
\\

\textbf{Пример.} Простая схема кодирования основанна на проверки четности.
Схема кодирования $E$ определяется следующим образом:
$$E:a_1 a_2...a_m \rightarrow b_1 b_2...b_m b_{m+1},$$
где,
\begin{center}

$b_i=a_i$ при $i\leq m$, \\
$b_{m+1}=1$ если $\sum\limits_{i=1}^m a_i$-нечетная, \\
$b_{m+1}=0$ иначе.

\end{center}