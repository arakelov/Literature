\documentclass[12pt,a5paper,oneside]{book}
\usepackage[utf8]{inputenc}
\usepackage[russian]{babel}
\usepackage[OT1]{fontenc}
\usepackage{amsmath}
\usepackage{amsfonts}
\usepackage{amssymb}
\usepackage{makeidx}
\usepackage{graphicx}
\usepackage[left=2cm,right=2cm,top=2cm,bottom=2cm]{geometry}
\author{Гурген Г. Аракелов}
\title{Архитектура UNIX}

\begin{document}
\maketitle
\tableofcontents

\chapter{Ядро и процессы}
\section{Режим пространство и контекст}
Для каждого процесса существуют два важных объекта:  \textit{область u} и \textit{стек ядра}.
\section{Процессы}
Процесс - это нечто, выполняющее экземпляр некоторой программы и создающее среду для ее функционирования
Большинство процессов создаются при помощи системного вызова \textit{fork} или \textit{vfork} и существуют до тех пор, пока не вызван \textit{exit}.
В процессе функционирования процесс может запускать одну или несколько программ, с помощью метода \textit{exec}. 
\end{document}